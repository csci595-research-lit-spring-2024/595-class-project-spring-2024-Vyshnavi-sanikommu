\chapter{Introduction}
\label{ch:into} % This how you label a chapter and the key (e.g., ch:into) will be used to refer this chapter ``Introduction'' later in the report. 
% the key ``ch:into'' can be used with command \ref{ch:intor} to refere this Chapter.
Diabetes is a fast growing disease among the people even in youngsters nowadays. It is necessary to understand how it develops in our body. Firstly, we need to understand how a body works without diabetes. Sugar comes from the food that we eat, specially carbohydrates. When we eat this food, the body breaks them down to Sugars or Glucose. This glucose moves around the body in the bloodstream. This glucose was needed by body parts like brain, pancreas to function. The remainder of glucose is taken to the cells of our body and liver which is stored as energy for later use. In order to use glucose for our body, insulin is required which is a hormone generated by pancreas. Insulin is a key to a closed door which helps glucose moves from blood stream. If pancreas is not able to produce enough insulin or if our body cannot use insulin it produces then glucose levels increases in bloodstream which leads to diabetes.

~\\[0cm]
According to World Health Organization, diabetes is major cause of death worldwide. Around 422 million people worldwide have diabetes. Indeed, it caused deaths of 2 million people in 2019.

\section*{Types of Diabetes}
\textbf{Type 1} diabetes appear most often during childhood and characterized by the partial functioning of pancreas. The cells fail to produce sufficient amounts of Insulin. Initially, we do not see any symptoms as the pancreas remains partially functional. There is no proven study and known methods for prevention.

~\\[0cm]
\textbf{Type 2} diabetes affects how the body uses sugars for energy. The cells produce low quantity of insulin or the body stops using insulin which can lead to high levels of sugar in bloodstream. High levels of glucose in the bloodstream and urine referred as Diabetes Mellitus. It is most common type of diabetes found in many people. It is caused by genetic factors and the lifestyle. It affects older adults and more obese or overweight people.

~\\[0cm]
Early prediction of the disease can be controlled and save the life. To accomplish this, this work explores prediction of diabetes by taking various attributes related to diabetes disease. For this purpose, we use the Diabetes Dataset \cite{dataset} which has 2000 instances with 9 attributes - `Pregnancies`, `Glucose`, `BloodPressure`, `SkinThickness`, `Insulin`, `BMI`, `DiabetesPedigreeFunction`, `Age`, `Outcome` for creating classification models using logistic regression and support vector machine algorithms.



%%%%%%%%%%%%%%%%%%%%%%%%%%%%%%%%%%%%%%%%%%%%%%%%%%%%%%%%%%%%%%%%%%%%%%%%%%%%%%%%%%%
\section{Background}
\label{sec:into_back}
This study is undertaken to address the crucial need for accurate and reliable methods in predicting diabetes using logistic regression and support vector machine algorithms on patient data. The motivation behind this study stems from the growing significance of leveraging machine learning algorithms to assist medical professionals in diagnosing and managing diabetes. Predictive models based on patient data offer the potential to identify individuals at risk or in the early stages of diabetes, enabling proactive and personalized healthcare strategies. Algorithms - logistic regression and support vector machine are chosen due to their widespread use in medical data analysis and their capability to handle classification tasks. The study outcomes have practical implications for healthcare practitioners and researchers, offering insights into algorithm selection for accurate and interpret diabetes prediction.

%%%%%%%%%%%%%%%%%%%%%%%%%%%%%%%%%%%%%%%%%%%%%%%%%%%%%%%%%%%%%%%%%%%%%%%%%%%%%%%%%%%
\section{Research question}
\label{sec:intro_prob_art, linewidth:80}
This section provides a detailed exploration of the research problem. The main question is about understanding how accurate logistic regression and support vector machine are in predicting diabetes from patient's data. The focus is on figuring out what factors cause differences in accuracy, especially looking at the impact of hyper parameters.

%%%%%%%%%%%%%%%%%%%%%%%%%%%%%%%%%%%%%%%%%%%%%%%%%%%%%%%%%%%%%%%%%%%%%%%%%%%%%%%%%%%
\section{Aims and objectives}
\label{sec:intro_aims_obj}

The primary aim of this study is to assess and compare the predictive accuracy of logistic regression and support vector machine algorithms based on diabetic patient's data to decide if a patient is diabetic or not. This study aims to provide valuable insights into the strengths and limitations of these supervised machine learning approaches, contributing to the enhancement of diabetes prediction methodologies.

\textbf{Objectives:} 
The main objectives of this study is to do data processing, implement logistic regression and support vector machine models, hyper parameter tuning to achieve highest possible predictive accuracy. At the end, we will evaluate the performance of both classification models.

%%%%%%%%%%%%%%%%%%%%%%%%%%%%%%%%%%%%%%%%%%%%%%%%%%%%%%%%%%%%%%%%%%%%%%%%%%%%%%%%%%%
\section{Solution approach}
\label{sec:intro_sol} % label of Org section
Briefly describe the solution approach and the methodology applied in solving the set aims and objectives.

Depending on the project, you may like to alter the ``heading'' of this section. Check with you supervisor. Also, check what subsection or any other section that can be added in or removed from this template.

\subsection{A subsection 1}
\label{sec:intro_some_sub1}
You may or may not need subsections here. Depending on your project's needs, add two or more subsection(s). A section takes at least two subsections. 

\subsection{A subsection 2}
\label{sec:intro_some_sub2}
Depending on your project's needs, add more section(s) and subsection(s).

\subsubsection{A subsection 1 of a subsection}
\label{sec:intro_some_subsub1}
The command \textbackslash subsubsection\{\} creates a paragraph heading in \LaTeX.

\subsubsection{A subsection 2 of a subsection}
\label{sec:intro_some_subsub2}
Write your text here...

%%%%%%%%%%%%%%%%%%%%%%%%%%%%%%%%%%%%%%%%%%%%%%%%%%%%%%%%%%%%%%%%%%%%%%%%%%%%%%%%%%%
\section{Summary of contributions and achievements} %  use this section 
\label{sec:intro_sum_results} % label of summary of results
Describe clearly what you have done/created/achieved and what the major results and their implications are. 
