\chapter{Introduction}
\label{ch:into} % This how you label a chapter and the key (e.g., ch:into) will be used to refer this chapter ``Introduction'' later in the report. 
% the key ``ch:into'' can be used with command \ref{ch:intor} to refere this Chapter.

Diabetes is a fast growing disease among the people even in youngsters nowadays. It is necessary to understand how it develops in our body. Firstly, we need to understand how a body works without diabetes. Sugar comes from the food that we eat, specially carbohydrates. When we eat this food, the body breaks them down to Sugars or Glucose. This glucose moves around the body in the bloodstream. This glucose was needed by body parts like brain, pancreas to function. The remainder of glucose is taken to the cells of our body and liver which is stored as energy for later use. In order to use glucose for our body, insulin is required which is a hormone generated by pancreas. Insulin is a key to a closed door which helps glucose moves from blood stream. If pancreas is not able to produce enough insulin or if our body cannot use insulin it produces then glucose levels increases in bloodstream which leads to diabetes. Diabetes is major cause of death in the world. Early prediction of the disease can be controlled and save the life. To accomplish this, this work explores prediction of diabetes by taking various attributes related to diabetes disease. For this purpose, we use the Diabetes Dataset -\cite{dataset} which has 2000 instances with 9 attributes - Pregnancies, Glucose, BloodPressure, SkinThickness, Insulin, BMI, DiabetesPedigreeFunction, Age, Outcome for creating classification models using Logistic Regression and Support Vector Machine algorithms.

%%%%%%%%%%%%%%%%%%%%%%%%%%%%%%%%%%%%%%%%%%%%%%%%%%%%%%%%%%%%%%%%%%%%%%%%%%%%%%%%%%%
\section{Background}
\label{sec:into_back}
This study is undertaken to address the crucial need for accurate and reliable methods in predicting diabetes using Logistic Regression and Support Vector Machine algorithms on patient data. The motivation behind this study stems from the growing significance of leveraging machine learning algorithms to assist medical professionals in diagnosing and managing diabetes. Predictive models based on patient data offer the potential to identify individuals at risk or in the early stages of diabetes, enabling proactive and personalized healthcare strategies. Algorithms - LR and SVM are chosen due to their widespread use in medical data analysis and their capability to handle classification tasks. The study outcomes have practical implications for healthcare practitioners and researchers, offering insights into algorithm selection for accurate and interpret diabetes prediction.

%%%%%%%%%%%%%%%%%%%%%%%%%%%%%%%%%%%%%%%%%%%%%%%%%%%%%%%%%%%%%%%%%%%%%%%%%%%%%%%%%%%
\section{Problem statement}
\label{sec:intro_prob_art}
This section provides a detailed exploration of the research problem. The main question is about understanding how accurate Logistic Regression and Support Vector Machine are in predicting diabetes from patient's data. The focus is on figuring out what factors cause differences in accuracy, especially looking at the impact of hyper parameters.

%%%%%%%%%%%%%%%%%%%%%%%%%%%%%%%%%%%%%%%%%%%%%%%%%%%%%%%%%%%%%%%%%%%%%%%%%%%%%%%%%%%
\section{Aims and objectives}
\label{sec:intro_aims_obj}

The primary aim of this study is to assess and compare the predictive accuracy of Logistic Regression and SVM algorithms based on diabetic patient's data to decide if a patient is diabetic or not. 

\textbf{Objectives:} The objectives are a set of tasks you would perform in order to achieve the defined aims. The objective statements have to be specif ic and measurable through the results and outcome of the project.



%%%%%%%%%%%%%%%%%%%%%%%%%%%%%%%%%%%%%%%%%%%%%%%%%%%%%%%%%%%%%%%%%%%%%%%%%%%%%%%%%%%
\section{Solution approach}
\label{sec:intro_sol} % label of Org section
Briefly describe the solution approach and the methodology applied in solving the set aims and objectives.

Depending on the project, you may like to alter the ``heading'' of this section. Check with you supervisor. Also, check what subsection or any other section that can be added in or removed from this template.

\subsection{A subsection 1}
\label{sec:intro_some_sub1}
You may or may not need subsections here. Depending on your project's needs, add two or more subsection(s). A section takes at least two subsections. 

\subsection{A subsection 2}
\label{sec:intro_some_sub2}
Depending on your project's needs, add more section(s) and subsection(s).

\subsubsection{A subsection 1 of a subsection}
\label{sec:intro_some_subsub1}
The command \textbackslash subsubsection\{\} creates a paragraph heading in \LaTeX.

\subsubsection{A subsection 2 of a subsection}
\label{sec:intro_some_subsub2}
Write your text here...

%%%%%%%%%%%%%%%%%%%%%%%%%%%%%%%%%%%%%%%%%%%%%%%%%%%%%%%%%%%%%%%%%%%%%%%%%%%%%%%%%%%
\section{Summary of contributions and achievements} %  use this section 
\label{sec:intro_sum_results} % label of summary of results
Describe clearly what you have done/created/achieved and what the major results and their implications are. 


%%%%%%%%%%%%%%%%%%%%%%%%%%%%%%%%%%%%%%%%%%%%%%%%%%%%%%%%%%%%%%%%%%%%%%%%%%%%%%%%%%%
\section{Organization of the report} %  use this section
\label{sec:intro_org} % label of Org section
Describe the outline of the rest of the report here. Let the reader know what to expect ahead in the report. Describe how you have organized your report. 

\textbf{Example: how to refer a chapter, section, subsection}. This report is organised into seven chapters. Chapter~\ref{ch:lit_rev} details the literature review of this project. In Section~\ref{ch:method}...  % and so on.

\textbf{Note:}  Take care of the word like ``Chapter,'' ``Section,'' ``Figure'' etc. before the \LaTeX command \textbackslash ref\{\}. Otherwise, a  sentence will be confusing. For example, In \ref{ch:lit_rev} literature review is described. In this sentence, the word ``Chapter'' is missing. Therefore, a reader would not know whether 2 is for a Chapter or a Section or a Figure.

