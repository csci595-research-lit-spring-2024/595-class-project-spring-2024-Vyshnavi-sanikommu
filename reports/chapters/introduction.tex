\chapter{Introduction}
\label{ch:into} % This how you label a chapter and the key (e.g., ch:into) will be used to refer this chapter ``Introduction'' later in the report. 
% the key ``ch:into'' can be used with command \ref{ch:intor} to refere this Chapter.
Diabetes is a fast growing disease among the people even in youngsters nowadays. It is necessary to understand how it develops in our body. Firstly, we need to understand how a body works without diabetes. Sugar comes from the food that we eat, specially carbohydrates. When we eat this food, the body breaks them down to Sugars or Glucose. This glucose moves around the body in the bloodstream. This glucose is needed by body parts like the brain and pancreas to function. The remainder of glucose is taken to the cells of our body and liver which is stored as energy for later use. In order to use glucose for our body, insulin is required which is a hormone generated by pancreas. Insulin is a key to a closed door which helps glucose moves from blood stream. If pancreas is not able to produce enough insulin or if our body cannot use insulin it produces then glucose levels increases in bloodstream which leads to diabetes.

According to World Health Organization, diabetes is major cause of death worldwide. Around 422 million people worldwide have diabetes. Indeed, it caused deaths of 2 million people in 2019.

There are two types of diabetes present as a disease in human beings: \textbf{Type 1} diabetes appear most often during childhood and is characterized by the partial functioning of the pancreas. The cells fail to produce sufficient amounts of Insulin. Initially, we do not see any symptoms as the pancreas remains partially functional. There is no proven study and known methods for prevention. \textbf{Type 2} diabetes affects how the body uses sugars for energy. The cells produce low quantity of insulin or the body stops using insulin which can lead to high levels of sugar in bloodstream. High levels of glucose in the bloodstream and urine referred as diabetes mellitus. It is most common type of diabetes found in many people. It is caused by genetic factors and the lifestyle. It affects older adults and more obese or overweight people.

Early prediciton of diabetes can help in controlling the disease and potentially save lives. To accomplish this, this 
work explores prediction of diabetes by taking various attributes related to diabetes disease. For this purpose, we use the diabetes dataset \citep{dataset} which has 2000 instances with 9 attributes - `Pregnancies`, `Glucose`, `BloodPressure`, `SkinThickness`, `Insulin`, `BMI`, `DiabetesPedigreeFunction`, `Age`, `Outcome` for creating classification models using logistic regression and support vector machine algorithms.



%%%%%%%%%%%%%%%%%%%%%%%%%%%%%%%%%%%%%%%%%%%%%%%%%%%%%%%%%%%%%%%%%%%%%%%%%%%%%%%%%%%
\section{Background}
\label{sec:into_back}
This study is undertaken to address the crucial need for accurate and reliable methods in predicting diabetes using logistic regression and support vector machine algorithms on patient data. The motivation behind this study stems from the growing significance of leveraging machine learning algorithms to assist medical professionals in diagnosing and managing diabetes. Predictive models based on patient data offer the potential to identify individuals at risk or in the early stages of diabetes, enabling proactive and personalized healthcare strategies. Algorithms - logistic regression and support vector machine are chosen due to their widespread use in medical data analysis and their capability to handle classification tasks. The study outcomes have practical implications for healthcare practitioners and researchers, offering insights into algorithm selection for accurate and interpret diabetes prediction.

%%%%%%%%%%%%%%%%%%%%%%%%%%%%%%%%%%%%%%%%%%%%%%%%%%%%%%%%%%%%%%%%%%%%%%%%%%%%%%%%%%%
\section{Research question}
\label{sec:intro_prob_art}
The research questions are: 
\begin{enumerate}
    \item How do the performance metrics of Logistic Regression (LR) and Support Vector Machines (SVM) models differ in predicting diabetes based on patient's data?
    \item What factors contribute to these differences, particularly in relation to the hyper parameters?  
    \item How do various meta parameter searches contribute to get the best performance from these two models?
\end{enumerate}

The research aims to compare the predictive performance metrics of logistic regression and support vector machine models for diabetes prediction based on patient data. It includes an in-depth analysis of the factors influencing the differences in performance metrics, with a specific focus on hyper parameters. Additionally, the study explores the impact of various meta-parameter searches on optimizing the overall performance of the models, aiming to identify the most effective parameters for diabetes prediction.
%%%%%%%%%%%%%%%%%%%%%%%%%%%%%%%%%%%%%%%%%%%%%%%%%%%%%%%%%%%%%%%%%%%%%%%%%%%%%%%%%%%
\section{Aims and objectives}
\label{sec:intro_aims_obj}

The primary aim of this study is to assess and compare the predictive performance metrics of logistic regression and support vector machine algorithms based on diabetic patient's data to decide if a patient is diabetic or not. This study aims to provide valuable insights into the strengths and limitations of these supervised machine learning approaches, contributing to the enhancement of diabetes prediction methodologies.

The main objectives of this study include:
\begin{itemize}
    \item Obtain the model data, clean and analyze it and prepare it for model training.
    \item  Train a standard logistic regression classification model on the labeled diabetes dataset.
    \item Train a competing SVM model on the same data.
    \item Explore model meta parameter and tune models to maximize predictive performance on accuracy, precision, recall, f1-score and confusion matrix.
    \item Evaluate the results of performance metrics after hyper parameter tuning to determine the best model for diabetes prediction. 
\end{itemize}


%%%%%%%%%%%%%%%%%%%%%%%%%%%%%%%%%%%%%%%%%%%%%%%%%%%%%%%%%%%%%%%%%%%%%%%%%%%%%%%%%%%
\section{Solution approach}
\label{sec:intro_sol} % label of Org section
The study follows a systematic approach encompassing model implementation, data preprocessing, splitting of data, hyper parameter tuning and performance analysis. Thorough exploration of various meta parameter search strategies contribute to achieving the aims. I will make sure to provide a clear documentation to make sure results are reliable and can be easily reproducible.

\subsection{Dataset Description}
\label{sec:intro_some_sub1}
This Diabetes Dataset \cite{dataset} has 2000 instances with 9 attributes - `Pregnancies`, `Glucose`, `BloodPressure`, `SkinThickness`, `Insulin`, `BMI`, `DiabetesPedigreeFunction`, `Age`, `Outcome`. The `Outcome` attribute indicates positive or negative for diabetes. 

\subsection{Data Preprocessing}
\label{sec:intro_some_sub2}
Data preprocessing is most important process. Mostly healthcare related data contains missing values that causes effectiveness of data. This process is essential for accurate result and successful predication using machine learning techniques.

\subsubsection{Missing values removal}
\label{sec:intro_some_subsub1}
This process is meant to identify instances with zero value and eliminate all such instances. Through eliminating irrelevant instances we make feature subset and this process is called feature subset selection which reduces the dimensionality of data.

\subsubsection{Splitting of data}
\label{sec:intro_some_subsub2}
After cleaning the data, the data is normalized in training and testing the models. After split, we train algorithm with training dataset and keep testing dataset aside. This testing dataset is used to test the trained model.

\subsection{Model Implementation}
\label{sec:intro_some_sub3}
After data is ready, we apply machine learning techniques. Implement logistic regression and support vector machine models to predict diabetes on the dataset.
\subsubsection{Logistic Regression}
Logistic Regression is one of the most common classification models. It is used for classification task where the goal is to predict the probability that an instance belongs to a given class or not. Create a standard logistic regression classification model, train the model with training dataset, and calculate the performance metrics. Similarly calculate the performance metrics for testing dataset. For this classifier, there are multiple hyper parameters such as regularization strength (C), solver, penalty.
\subsubsection{Support Vector Machine}
Support Vector Machine is a powerful machine learning algorithm used for linear or nonlinear classification, regression, and outlier detection tasks. This classifier aims to establish a hyperplane that can separate the classes by adjusting the distance between data points and the hyperplane. 


\subsection{Hyper Parameter Tuning and Performance Analysis}
\label{sec:intro_some_sub4}
To determine the best performance model, we should consider the factors like hyper parameters, meta parameter search strategies for logistic regression and support vector machine to achieve more predictive performance metrics.
\subsubsection{Grid Search}
Grid Search is a method for hyper parameter optimization that involves specifying a list of values for each hyper parameter to optimize. Subsequently, the model is trained for each combination of these values, and the optimal values for the hyper parameters are selected based on the models' performance.
For logistic regression classifier, there are multiple hyper parameters such as regularization strength (C), solver, penalty. For support vector machine classifier, there are multiple hyper parameters such as regularization parameter (C), kernel, gamma, degree.

%%%%%%%%%%%%%%%%%%%%%%%%%%%%%%%%%%%%%%%%%%%%%%%%%%%%%%%%%%%%%%%%%%%%%%%%%%%%%%%%%%%
\section{Summary of contributions and achievements} %  use this section 
\label{sec:intro_sum_results} % label of summary of results
This research focused on evaluating the performance of logistic regression and support vector machine models in predicting diabetes, examining the impact of hyperparameter tuning on these models. The primary objective was to compare the baseline performance of these models and then determine whether hyperparameter tuning could significantly improve their predictive capabilities.

Both models exhibited good performance on the training and testing dataset, while svm slightly outperformed lr on the test dataset, achieving 82\% accuracy while lr achieved 79\%. After applying hyperparameter tuning with gridsearchCV, svm achieved a training accuracy of 85\% and testing accuracy of 80\%, demonstrating the significant impact of tuning on model performance.

The results of this study have several implications:
\begin{itemize}
    \item Effectiveness of Hyperparameter Tuning: The improved accuracy and other metrics demonstrate that hyperparameter tuning can lead to significant performance gains, making it a crucial step in machine learning model development.
    \item Potential for SVM in Diabetes Prediction: Given its superior performance, the svm model shows promise for further research and deployment in diabetes prediction applications.
\end{itemize}

Overall, this research has demonstrated that both models can effectively predict diabetes, and that hyperparameter tuning can significantly enhance their performance. Further research could explore additional tuning strategies, other machine learning models, or ensemble methods to improve prediction accuracy and reliability.

