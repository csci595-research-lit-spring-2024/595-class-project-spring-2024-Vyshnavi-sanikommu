\chapter{Literature Review}
\label{ch:lit_rev} %Label of the chapter lit rev. The key ``ch:lit_rev'' can be used with command \ref{ch:lit_rev} to refer this Chapter.
The examination of relevant studies yields findings from various healthcare datasets, where researchers conducted analyses and predictions employing a range of methods and techniques. Numerous prediction models have been devised and applied by various researchers, utilizing different forms of data mining techniques, machine learning algorithms, or a combination of these methodologies.
% PLEAE CHANGE THE TITLE of this section
\section{Review of state-of-the-art} 
% Note the use of \cite{} and \citep{}
\cite{mujumdar2019diabetes} aims to create a system using machine learning algorithm and deep learning techniques to provide accurate results and reduce human efforts. The diabetes dataset contains 800 instances with 10 attributes. This study implemented various machine learning algorithms include Support Vector Classifier, Random Forest Classifier, Decision Tree Classifier, Extra Tree Classifier, Ada Boost algorithm, Perceptron, Linear Discriminant Analysis algorithm, Logistic Regression, K-Nearest Neighbour, Gaussian Naïve Bayes,Bagging algorithm, Gradient Boost Classifier. The study incorporates the concept of pipelining and compares the diabetes dataset with the Pima dataset. Performance analysis includes metrics like classification accuracy, confusion matrix, f1-score, precision, and recall. The findings reveal that Logistic Regression achieves the highest accuracy of 96\%, indicating an improvement in accuracy for the diabetes dataset compared to the Pima diabetes dataset. The study concludes that implementing a pipeline model enhances the accuracy of the classification performance, with the Ada Booster classifier identified as the best model, achieving an accuracy of 98.8\%.

\cite{soni2020diabetes} aims to design and implement Diabetes prediction using machine learning methods and performance analysis of that methods for early prediction and to cure diabetes and save humans life. The diabetes dataset is gathered from  UCI repository which is named as Pima Indian Diabetes dataset. The dataset have many attributes of 768 patients. The proposed methodology involves the utilization of various classification and ensemble learning methods, including SVM, Logistic Regression, KNN, Rndom Forest, Decision Tree, Gradient Boosting classifiers are used. The findings indicate a 77\% accuracy achieved through an 80:20 split. The study concludes that Random Forest classifier exhibits highest accuracy when compared to other machine learning methods.

\cite{swapna2018diabetes} aims to develop a methodology for classification of diabetic and normal Heart rate variability (HRV) signals using advanced deep learning architectures, specifically employing Long Short-Term Memory (LSTM), Convolutional Neural Network (CNN), and combinations. The extracted features are then passed into a Support Vector Machine (SVM) for accurate classification. The study demonstrates performance improvements in CNN and CNN-LSTM architectures compared to earlier work, achieving a high accuracy of 95.7\%. 

% A possible section of you chapter
\section{Critique of the review} % Use this section title or choose a betterone
The literature review provides insights related to diabetes prediction using machine learning and deep learning techniques. Each study aims to contribute to the development of reliable and accurate models for diabetes diagnosis, with a focus on enhancing classification performance. \cite{mujumdar2019diabetes} focus on creating an efficient system, achieving a 96\% accuracy with Logistic Regression and identifying Ada Booster as the best model. \cite{soni2020diabetes} design a predictive system, achieving 77\% accuracy with Random Forest being the most accurate. \cite{swapna2018diabetes} explore advanced deep learning architectures, obtaining a high accuracy of 95.7\% with CNN-LSTM and SVM. The key findings include the effectiveness of pipeling models, ensemble methods and the importance of accurate diabetes prediction for early intervention and patient care. However, the identified studies exhibit some limitations, such as a lack of detailed explanations regarding the selection of specific algorithms in the ensemble and a deficiency in exploring deep learning techniques that could enhance performance. Additionally, there is limited discussion on the generalizability of the proposed methodology to diverse datasets.Future research directions involve anomaly prediction and the utilization of larger datasets. \cite{yudheksha2022machine} presented a machine learning-based approach for early-stage diabetes prediction using a dataset of patient attributes. They demonstrated the effectiveness of their model in predicting diabetes risk, achieving promising results. Additionally, \cite{larabi2019current} conducted a comprehensive review of current techniques for diabetes prediction. Their review provided insights into various methods and strategies employed in diabetes prediction research, highlighting the importance of accurate prediction models in healthcare applications.
~\\

% Pleae use this section
\section{Summary} 
This literature review extensively examines the current research landscape in the realm of diabetes prediction, highlighting numerous technological advancements in this field. The literature review thoroughly investigates three prominent studies in diabetes prediction using machine learning and deep learning. \cite{mujumdar2019diabetes} employ innovative pipelining techniques for accuracy enhancement but lack in-depth exploration of neural networks. \cite{soni2020diabetes} focus on diverse classifiers without extensive rationale, neglecting potential gains from deep learning. \cite{swapna2018diabetes} \cite {kamble2016diabetes} excel in deep learning, emphasizing the need for a unified approach and comprehensive evaluation. The critiques call for collaboration, integration of machine learning strengths, and standardized methodologies for future research in diabetes prediction.
~\\
