\chapter{Literature Review}
\label{ch:lit_rev} %Label of the chapter lit rev. The key ``ch:lit_rev'' can be used with command \ref{ch:lit_rev} to refer this Chapter.
The examination of relevant studies yields findings from various healthcare datasets, where researchers conducted analyses and predictions employing a range of methods and techniques. Numerous prediction models have been devised and applied by various researchers, utilizing different forms of data mining techniques, machine learning algorithms, or a combination of these methodologies.
% PLEAE CHANGE THE TITLE of this section
\section{Review of state-of-the-art} 
% Note the use of \cite{} and \citep{}
\cite{mujumdar2019diabetes} aims to create a system using machine learning algorithm and deep learning techniques to provide accurate results and reduce human efforts. The diabetes dataset contains 800 instances with 10 attributes. This study implemented various machine learning algorithms include Support Vector Classifier, Random Forest Classifier, Decision Tree Classifier, Extra Tree Classifier, Ada Boost algorithm, Perceptron, Linear Discriminant Analysis algorithm, Logistic Regression, K-Nearest Neighbour, Gaussian Naïve Bayes,Bagging algorithm, Gradient Boost Classifier. The study incorporates the concept of pipelining and compares the diabetes dataset with the Pima dataset. Performance analysis includes metrics like classification accuracy, confusion matrix, f1-score, precision, and recall. The findings reveal that Logistic Regression achieves the highest accuracy of 96\%, indicating an improvement in accuracy for the diabetes dataset compared to the Pima diabetes dataset. The study concludes that implementing a pipeline model enhances the accuracy of the classification performance, with the Ada Booster classifier identified as the best model, achieving an accuracy of 98.8\%.

\cite{soni2020diabetes} aims to design and implement Diabetes prediction using machine learning methods and performance analysis of that methods for early prediction and to cure diabetes and save humans life. The diabetes dataset is gathered from  UCI repository which is named as Pima Indian Diabetes dataset. The dataset have many attributes of 768 patients. The proposed methodology involves the utilization of various classification and ensemble learning methods, including SVM, Logistic Regression, KNN, Rndom Forest, Decision Tree, Gradient Boosting classifiers are used. The findings indicate a 77\% accuracy achieved through an 80:20 split. The study concludes that Random Forest classifier exhibits highest accuracy when compared to other machine learning methods.

These are the examples of how to \textit{cite} external sources, seminal works, and research papers. In \LaTeX, if you use ``\textbf{BibTex}'' you do not have to worry much since the proper use of a bibliographystyle package like ``agsm for the Harvard style'' and little rectification of the content in a BiBText source file [In this template, BibTex are stored in the ``references.bib'' file], we can conveniently generate  a reference style. 

Take a note of the commands \textbackslash cite\{\} and \textbackslash citep\{\}. The command \textbackslash cite\{\} will write like ``Author et al. (2019)'' style for Harvard, APA and Chicago style. The command \textbackslash citep\{\} will write like ``(Author et al., 2019).'' Depending on how you construct a sentence, you need to use them smartly. Check the examples of \textbf{in-text citation} of sources listed here [This template recommends the \textbf{Harvard style} of referencing.]:
\begin{itemize}
    \item \cite{lamport1994latex} has written a comprehensive guide on writing in \LaTeX ~[Example of \textbackslash cite\{\} ].
    \item If \LaTeX~is used efficiently and effectively, it helps in writing a very high-quality project report~\citep{lamport1994latex} ~[Example of \textbackslash citep\{\} ].   
    \item A detailed APA, Harvard, and Chicago referencing style guide are available in~\citep{uor_refernce_style}.
\end{itemize}

\noindent 
Example of a numbered list:
\begin{enumerate}
    \item \cite{lamport1994latex} has written a comprehensive guide on writing in \LaTeX.
    \item If \LaTeX is used efficiently and effectively, it helps in writing a very high-quality project report~\citep{lamport1994latex}.   
\end{enumerate}

% PLEAE CHANGE THE TITLE of this section
\section{Project Description in context of Existing Literature}
Using other sources, ideas, and material always bring with it a risk of unintentional plagiarism. 

\noindent
\textbf{\color{red}MUST}: do read the university guidelines on the definition of plagiarism as well as the guidelines on how to avoid plagiarism~\citep{uor_plagiarism}.




% A possible section of you chapter
\section{Critique of the review} % Use this section title or choose a betterone
Describe your main findings and evaluation of the literature. ~\\

% Pleae use this section
\section{Summary} 
Write a summary of this chapter~\\
