\chapter{Reflection}
\label{ch:reflection}
Throughout this project, I gained significant skills in technical problem-solving. The diabetes prediction work involved diving into machine learning, experimenting with logistic regression and support vector machine algorithms, and mastering data processing. Although I had some prior knowledge, the in-depth exploration of model evaluation, optimization, and tuning proved to be an invaluable learning experience.

A crucial skill I developed during this project was the ability to systematically tackle complex problems. My initial plan was straightforward: gather data, preprocess it, and apply machine learning to predict diabetes. However, I faced unexpected challenges, including selecting performance metrics, choosing features, and tuning hyperparameters. These hurdles led me to adopt an iterative approach, emphasizing continuous improvement and validating each step to ensure the best outcomes.

This iterative process enhanced my skills in experimenting with lr and svm algorithms, comparing their performance, and fine-tuning them for better results. Learning to use gridsearchCV for hyperparameter tuning was a key takeaway—it showed me how even small parameter changes could greatly impact model accuracy and generalization.

Despite these successes, I faced challenges I couldn't fully resolve. Data imbalance and balancing precision with recall while keeping high accuracy were particularly tricky. I learned that perfect accuracy is often unattainable and that choosing the right model involves trade-offs. This experience deepened my understanding of machine learning and highlighted the need to evaluate models from multiple perspectives.

If I could redo this project, I would start with a thorough examination of the data, addressing imbalance earlier, and consider a wider variety of algorithms. I would also implement more rigorous cross-validation techniques from the outset to ensure my results were reliable and not influenced by random data fluctuations.

One deviation from my initial plan was the extended focus on hyperparameter tuning. Initially, I underestimated the impact it could have on model performance. As I progressed, I realized that tuning was essential for achieving optimal results. This shift in focus taught me the importance of flexibility and adaptability in research.

Overall, this project was a profound learning experience that enhanced my technical skills and problem-solving abilities. It also provided insights into the complexities of machine learning and the need for careful planning and adaptation. The lessons learned will undoubtedly influence my future work and approach to problem-solving in a broader context.


