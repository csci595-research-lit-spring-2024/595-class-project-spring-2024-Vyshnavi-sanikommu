\chapter{Conclusions and Future Work}
\label{ch:con}
\section{Conclusions}
Early detection of diabetes is one of the significant challenges in the health care industry. Our research delved into the realm of diabetes prediction using logistic regression and support vector machines models, aiming to shed light on their comparative performance metrics and underlying factors. 

In our research, we designed a system which can predict diabetes with high accuracy. We preprocessed the data by addressing missing values, zero-valued features, and employing imputation techniques across all features. Using the feature reduction method, we dropped two features. We used six input features - `Pregnancies`, `Glucose`, `SkinThickness`, `Insulin`, `BMI` and `Age` and one output feature `Outcome` in the  dataset \citep{dataset}. We explored the efficacy of two machine learning algorithms, lr and svm to predict diabetes and evaluated the performance on various measures like accuracy, precision, recall, and f1-score. These models provided an accuracy greater than 70\%. 

Furthermore, our exploration of hyperparameter optimization strategies underscores the importance of fine-tuning model parameters to achieve optimal predictive performance. To enhance model performance, we employed hyperparameter tunning using gridsearchcv strategy, optimizing parameters. These models provided almost same accuracy before tuning and svm outperformed LR for both train/test split method. After tuning, the optimized models exhibited notable improvements in accuracy, with svm achieving an impressive accuracy of approximately 80\% on the test dataset. while lr slightly trailing behind, still demonstrated commendable performance, boasting an accuracy close to 79\%.

\section{Future work}
This section should refer to Chapter~\ref{ch:results} where the author has reflected their criticality about their own solution. The future work is then sensibly proposed in this section.

\textbf{Guidance on writing future work:} While working on a project, you gain experience and learn the potential of your project and its future works. Discuss the future work of the project in technical terms. This has to be based on what has not been yet achieved in comparison to what you had initially planned and what you have learned from the project. Describe to a reader what future work(s) can be started from the things you have completed. This includes identifying what has not been achieved and what could be achieved. 



A good future work summary could be approximately 300--500 words long, but this is just a recommendation.