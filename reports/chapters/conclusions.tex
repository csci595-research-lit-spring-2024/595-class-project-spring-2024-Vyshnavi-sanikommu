\chapter{Conclusions and Future Work}
\label{ch:con}
\section{Conclusions}
Early detection of diabetes is one of the significant challenges in the health care industry. Our research delved into the realm of diabetes prediction using logistic regression and support vector machines models, aiming to shed light on their comparative performance metrics and underlying factors. 

In our research, we designed a system which can predict diabetes with high accuracy. We preprocessed the data by addressing missing values, zero-valued features, and employing imputation techniques across all features. Using the feature reduction method, we dropped two features. We used six input features - `Pregnancies`, `Glucose`, `SkinThickness`, `Insulin`, `BMI` and `Age` and one output feature `Outcome` in the  dataset \citep{dataset}. We explored the efficacy of two machine learning algorithms, lr and svm to predict diabetes and evaluated the performance on various measures like accuracy, precision, recall, and f1-score. These models provided an accuracy greater than 70\%.

Furthermore, our exploration of hyperparameter optimization strategies underscores the importance of fine-tuning model parameters to achieve optimal predictive performance. To enhance model performance, we employed hyperparameter tunning using gridsearchCV strategy, optimizing parameters. These models provided good accuracy before tuning and svm outperformed lr for both train/test split method. After tuning, the optimized models exhibited notable improvements in accuracy, with svm achieving an impressive accuracy of approximately 82\% on the test dataset. while lr slightly trailing behind, still demonstrated commendable performance, boasting an accuracy close to 81\%.

\section{Future work}
The results from Chapter~\ref{ch:results} indicate that both logistic regression and support vector machine classifiers achieved reasonable performance in predicting diabetes, with svm attaining slightly higher accuracy. However, there are several areas for future research that can further enhance the predictive capabilities of these models and address their limitations.

\begin{itemize}
\item A critical aspect for future work is the expansion of the dataset. The current study dataset may not fully represent the diverse population affected by diabetes. Increasing the dataset's size and diversity would improve the model's generalizability and robustness. It would also help reduce biases and ensure that the models work across different demographics and geographic locations.

\item Future studies can delve into additional feature engineering techniques. By incorporating domain-specific knowledge from medical professionals and using automated feature selection methods like recursive feature elimination, new features may be uncovered that contribute significantly to prediction accuracy. This approach could also lead to more refined models with reduced risks of overfitting.

\item Future work could explore more complex architectures such as deep neural networks (DNNs), convolutional neural networks (CNNs), and recurrent neural networks (RNNs). These models might capture intricate patterns in the data that traditional machine learning models might miss. Deep learning methods could improve prediction accuracy and add robustness to the models.

\item Another crucial aspect for future work is managing class imbalances. Class imbalance in the dataset can lead to biased models and high false negatives. Techniques like oversampling, undersampling, and Synthetic Minority Over-sampling Technique (SMOTE) could be used to balance the classes, resulting in improved recall and better model performance.

\item The current study utilized gridsearchCV for hyperparameter tuning, other methods such as randomizedsearchcv and bayesian optimization offer more flexible and efficient tuning processes. Future work could investigate these methods to optimize model performance further, potentially achieving higher accuracy and reliability.

\item Integrating these models into clinical practice is a valuable future direction. Building a user-friendly interface for healthcare professionals and conducting clinical trials would be essential to validate the models' effectiveness in real-world. This step is crucial for demonstrating that these models can assist in early diabetes diagnosis and improve patient outcomes.
\end{itemize}
