\chapter{Results}
\label{ch:results}
\section{Results for ML methods - LR and SVM}
The evaluation of machine learning algorithms is crucial for assessing their effectiveness in predicting diabetes. We use various performance metrics such as accuracy, precision, recall and f1-score to compare the classification methods. These metrics are calculated using equations \ref{eq:accuracy_formula}, \ref{eq:precision}, \ref{eq:recall}, \ref{eq:f1-score}, derived from the confusion matrix.

Table \ref{tab:p_metrics} shows performance measures of logistic regression and svm classifiers for training and testing datasets.

\begin{table}[h!]
    \centering
    \caption{Performance metrics}
    \label{tab:p_metrics}
    \begin{tabular}{llrrrr}     
        \toprule
        Classifier  &   DataSet Type &   Precision   &   Recall  &   F1-score    &   Accuracy (\%) \\
        \midrule
        LR  &   Training    &   0.70   &   0.55   &   0.62    &   76\%   \\
        LR  &   Testing &   0.74   &   0.58   &   0.65    &   78\%   \\
        SVM  &   Training    &   0.69   &   0.55   &   0.62    &   76\%   \\
        SVM  &   Testing &   0.75   &   0.58   &   0.66    &   79\%   \\
        \bottomrule
    \end{tabular}
\end{table}

Here, svm model achieves the highest accuracy of 79\% on test dataset, while both models exhibit similar accuracy on the training dataset. The confusion matrix for the SVM model on the test dataset is provided in Table \ref{tab:conf_matrix}.

\begin{table}[h!]
    \centering
    \caption{Confusion Matrix for SVM (Test Data)}
    \label{tab:conf_matrix}
    \begin{tabular}{llr}     
        \toprule
        & Diabetic  &   Non-diabetic \\
        \midrule
        Diabetic &  237  &   26   \\
        Non-diabetic &  57    &   80  \\
        \bottomrule
    \end{tabular}
\end{table}

Despite these promising results, determining the best model for predicting diabetes requires further investigation and analysis.

\section{Hyperparameter Tuning Results for LR and SVM Using GridSearchCV}
After applying GridSearchCV for hyperparameter tuning, both logistic regression and svm models demonstrated significantly improved performance compared to their standard configurations which was shown in Table \ref{tab:p_metrics}. 

Table \ref{tab:gs_p_metrics} shows performance measures of logistic regression and svm classifier for training and testing datasets after tuning.

\begin{table}[h!]
    \centering
    \caption{Performance metrics after hyperparameter tuning}
    \label{tab:gs_p_metrics}
    \begin{tabular}{llrrrr}     
        \toprule
        Classifier  &   DataSet Type &   Precision   &   Recall  &   F1-score    &   Accuracy (\%) \\
        \midrule
        LR  &   Training    &   0.70   &   0.56   &   0.62    &   77\%   \\
        LR  &   Testing &   0.75   &   0.58   &   0.66    &   79\%   \\
        SVM  &   Training    &   0.75   &   0.57   &   0.65    &   79\%   \\
        SVM  &   Testing &   0.80   &   0.56   &   0.66    &   80\%   \\
        \bottomrule
    \end{tabular}
\end{table}

Here, svm model achieves the highest accuracy of 80\% on the test dataset, while lr and svm models achieve 77\% and 79\% accuracy on the training dataset. The confusion matrix for the svm model on the test dataset is provided in Table \ref{tab:gs_conf_matrix}

\begin{table}[h!]
    \centering
    \caption{Confusion Matrix for SVM (Test Data)}
    \label{tab:gs_conf_matrix}
    \begin{tabular}{llr}     
        \toprule
        & Diabetic  &   Non-diabetic \\
        \midrule
        Diabetic &  244  &   19   \\
        Non-diabetic &  60    &   77  \\
        \bottomrule
    \end{tabular}
\end{table}

The optimized models achieved higher accuracy scores, highlighting the effectiveness of hyperparameter tuning in enhancing predictive capabilities.

\section{Summary}
The evaluation of machine learning algorithms lr and svm for predicting diabetes demonstrated promising results. While both models exhibited similar performance on the training dataset, svm outperformed lr slightly on the testing dataset with the highest accuracy of 79\%. Hyperparameter tuning using gridsearchcv further enhanced the models performance, with the optimized svm achieving a testing accuracy of 80\%. These findings underscore the effectiveness of hyperparameter tuning in improving predictive capabilities.



