\chapter{Results}
\label{ch:results}
\section{Results for ML methods - LR and SVM}
The evaluation of machine learning algorithms is crucial for assessing their effectiveness in predicting diabetes. We use various performance metrics such as accuracy, precision, recall and f1-score to compare the classification methods. These metrics are calculated using equations \ref{eq:accuracy_formula}, \ref{eq:precision}, \ref{eq:recall}, \ref{eq:f1-score}, derived from the confusion matrix.

Table \ref{tab:p_metrics} shows performance measures of logistic regression and svm classifiers for training and testing datasets.

\begin{table}[h!]
    \centering
    \caption{Performance metrics}
    \label{tab:p_metrics}
    \begin{tabular}{llrrrr}     
        \toprule
        Classifier  &   DataSet Type &   Precision   &   Recall  &   F1-score    &   Accuracy (\%) \\
        \midrule
        LR  &   Training    &   0.70   &   0.56   &   0.62    &   77\%   \\
        LR  &   Testing &   0.75   &   0.58   &   0.65    &   79\%   \\
        SVM  &   Training    &   0.76   &   0.71   &   0.74    &   83\%   \\
        SVM  &   Testing &   0.77   &   0.66   &   0.71    &   82\%   \\
        \bottomrule
    \end{tabular}
\end{table}

Here, lr model performed adequately on the training and testing datasets while svm model achieved higher accuracy, with 83\% on the training set and 82\% on the testing set. This higher accuracy, along with better precision, recall, and F1-score, suggests that svm might be a more suitable model for diabetes prediction. The confusion matrix for the svm model on the test dataset is provided in Table \ref{tab:conf_matrix}.

\begin{table}[h!]
    \centering
    \caption{Confusion Matrix for SVM (Test Data)}
    \label{tab:conf_matrix}
    \begin{tabular}{llr}     
        \toprule
        & Diabetic  &   Non-diabetic \\
        \midrule
        Diabetic &  236  &   27   \\
        Non-diabetic &  46    &   91  \\
        \bottomrule
    \end{tabular}
\end{table}

Despite these promising results, determining the best model for predicting diabetes requires further investigation and analysis.

\section{Hyperparameter Tuning Results for LR and SVM Using GridSearchCV}
After applying gridsearchCV for hyperparameter tuning, both logistic regression and svm models demonstrated significantly improved performance compared to their standard configurations which was shown in Table \ref{tab:p_metrics}. 

Table \ref{tab:gs_p_metrics} shows performance measures of logistic regression and svm classifier for training and testing datasets after tuning.

\begin{table}[h!]
    \centering
    \caption{Performance metrics after hyperparameter tuning}
    \label{tab:gs_p_metrics}
    \begin{tabular}{llrrrr}     
        \toprule
        Classifier  &   DataSet Type &   Precision   &   Recall  &   F1-score    &   Accuracy (\%) \\
        \midrule
        LR  &   Training    &   0.72   &   0.54   &   0.62    &   77\%   \\
        LR  &   Testing &   0.80   &   0.57   &   0.67    &   81\%   \\
        SVM  &   Training    &   0.79   &   0.76   &   0.77    &   85\%   \\
        SVM  &   Testing &   0.78   &   0.66   &   0.72    &   82\%   \\
        \bottomrule
    \end{tabular}
\end{table}

Hyperparameter tuning significantly improved both models' performance. Here, svm model achieved the highest accuracy 85\% on the training dataset and 82\% on the testing dataset. LR model also saw improvements, with 81\% testing accuracy. While both models showed good precision, the recall could still be enhanced. Fine-tuning hyperparameters led to improved overall accuracy and balanced performance. However, further refinement may be needed to increase recall and reduce misclassifications. The confusion matrix for the svm model on the test dataset is provided in Table \ref{tab:gs_conf_matrix}

\begin{table}[h!]
    \centering
    \caption{Confusion Matrix for SVM (Test Data)}
    \label{tab:gs_conf_matrix}
    \begin{tabular}{llr}     
        \toprule
        & Diabetic  &   Non-diabetic \\
        \midrule
        Diabetic &  238  &   25   \\
        Non-diabetic &  46    &   91  \\
        \bottomrule
    \end{tabular}
\end{table}

The optimized models achieved higher accuracy scores, highlighting the effectiveness of hyperparameter tuning in enhancing predictive capabilities.

\section{Summary}
The evaluation of machine learning algorithms lr and svm for predicting diabetes demonstrated promising results. Although both models had comparable performance on the training dataset, svm outperformed lr on the testing dataset, achieving 82\% accuracy. Hyperparameter tuning with gridsearchCV further improved the models' performance, with the optimized svm achieving an accuracy of 82\% on the testing set, highlighting the effectiveness of this technique. These results suggest that svm might be more effective for diabetes prediction, but additional tuning and exploration could further enhance the performance of both models. These findings underscore the effectiveness of hyperparameter tuning in improving predictive capabilities.



