%Two resources useful for abstract writing.
% Guidance of how to write an abstract/summary provided by Nature: https://cbs.umn.edu/sites/cbs.umn.edu/files/public/downloads/Annotated_Nature_abstract.pdf %https://writingcenter.gmu.edu/guides/writing-an-abstract
\chapter*{\center \Large  Abstract}
%%%%%%%%%%%%%%%%%%%%%%%%%%%%%%%%%%%%%%
% Replace all text with your text
%%%%%%%%%%%%%%%%%%%%%%%%%%%%%%%%%%%

Diabetes is one of the most lethal diseases in the world. According to International Diabetes Foundation 537 million people are living with diabetes across the world. By 2045, this would be 783 million. Diabetes is a disease caused due to an increase in blood glucose level. High blood glucose produces the symptoms of frequent urination, increased hunger, and thirst. Diabetes is a leading cause of blindness, kidney failure, amputations, heart failure and stroke. When we eat, our body converts the food into sugars, or glucose. The pancreas releases insulin, a crucial hormone that unlocks cells to facilitate glucose entry. This mechanism enables our cells to utilize glucose as an energy source, supporting essential bodily functions. Machine learning is an emerging scientific field in data science dealing with the ways in which machines learn from experience. The aim of this project is to develop a system which can perform early prediction of diabetes for a patient with a higher accuracy by combining the results of supervised machine learning algorithms like Logistic Regression and Support vector machine. Machine learning techniques provide better results for prediction by constructing models from datasets collected from patients. The accuracy of the model using each of the algorithms is calculated by considering the meta parameters to achieve the best performance. Then the one with a good accuracy is taken as the model for predicting diabetes.



%%%%%%%%%%%%%%%%%%%%%%%%%%%%%%%%%%%%%%%%%%%%%%%%%%%%%%%%%%%%%%%%%%%%%%%%%s
~\\[1cm]
\noindent % Provide your key words
\textbf{Keywords:} Diabetes, Machine Learning, Logistic Regression, Support Vector Machine, Accuracy

\vfill
\noindent
\textbf{Report's total word count:} we expect a maximum of 10,000 words (excluding reference and appendices) and about 10 pages. [A good project report can also be written in approximately 5,000 words.]

