%Two resources useful for abstract writing.
% Guidance of how to write an abstract/summary provided by Nature: https://cbs.umn.edu/sites/cbs.umn.edu/files/public/downloads/Annotated_Nature_abstract.pdf %https://writingcenter.gmu.edu/guides/writing-an-abstract
\chapter*{\center \Large  Abstract}
%%%%%%%%%%%%%%%%%%%%%%%%%%%%%%%%%%%%%%
% Replace all text with your text
%%%%%%%%%%%%%%%%%%%%%%%%%%%%%%%%%%%

Diabetes is one of the most lethal diseases affecting 537 million people worldwide, is projected to rise to 783 million by 2045. Diabetes is a disease caused due to an increase in blood glucose level, causing symptoms like frequent urination, increased hunger, and thirst. Diabetes is a leading cause of blindness, kidney failure, amputations, heart failure and stroke. The body's conversion of food into glucose requires insulin, released by the pancreas, which unlocks cells for glucose entry. This process allows cells to use glucose as an energy source, supporting vital bodily functions. The aim of this project is to develop a system which can perform early prediction of diabetes for a patient, employing supervised machine learning algorithms such as logistic regression and support vector machine. Machine learning techniques provide better results for prediction by constructing models from datasets collected from patients. The project will focus on optimizing performance metrics through a systematic search and tuning of model hyper parameters to maximize overall model effectiveness. The evaluation of model performance will encompass accuracy, precision, recall, f1-score, and confusion matrix. We will compare the performance metrics of the base model with those obtained after tuning using grid search. The results indicate that grid search provides optimal hyper parameters, contributing to the determination of the best-performing model.



%%%%%%%%%%%%%%%%%%%%%%%%%%%%%%%%%%%%%%%%%%%%%%%%%%%%%%%%%%%%%%%%%%%%%%%%%s
~\\[1cm]
\noindent % Provide your key words
\textbf{Keywords:} diabetes, machine learning, logistic regression, support vector machine, performance metrics

